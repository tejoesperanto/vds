%
% Specifiko pri la reta vocxdonsistemo de TEJO.
%
% Teksto skribita de Pauxlo Ebermann.
%

\documentclass[draft]{scrartcl}

% esperanta teksto.
\usepackage[esperanto]{babel}


\usepackage[utf8x]{inputenc}
\usepackage{lmodern}

% Klakeblaj ligoj ene de la dokumento
\usepackage[final]{hyperref}

\newenvironment{itemize*}{%
  \begin{itemize}%
  \setlength{\itemsep}{0pt}%
  \setlength{\parsep}{0pt}%
  \setlength{\parskip}{2pt plus 1pt}
  \setlength{\topsep}{1pt}%
}{%
    \end{itemize}%
}

\newenvironment{enumerate*}{%
  \begin{enumerate}%
  \setlength{\itemsep}{0pt}%
  \setlength{\parsep}{0pt}%
  \setlength{\parskip}{2pt plus 1pt}
  \setlength{\topsep}{1pt}%
  \setlength{\partopsep}{0pt}%
}{%
    \end{enumerate}%
}


\begin{document}

\title{Specifo pri la reta voĉdonsistemo de TEJO}
\author{Paŭlo Ebermann}

\maketitle

\vspace*{-\baselineskip}

\begin{abstract}
  Detaloj de funkciado de la Reta Voĉdonsistemo de TEJO
  el vidpunkto de regularo/reglamento kaj uzantoj.

  La programo nun (4a de septembro 2010) funkcias tiel, kiel
  estas priskribita en tiu ĉi dokumento (krom eblaj cimoj).
\end{abstract}
\enlargethispage{2\baselineskip}
\vspace*{-\baselineskip}

\tableofcontents

\section{Juraj bazoj}

\subsection{El la Komitata Reglamento: §5. Reta voĉdono}\label{reglamento}


\begin{enumerate*}

\item Ĉiu komitatano rajtas postuli retan voĉdonon pri iu ajn decidota
punkto. Se la Komitato estas kunvenanta, reta voĉdono okazas se la
Komitato akceptas la postulon. Se la Komitato ne estas kunvenanta,
reta voĉdono aŭtomate okazas.

\item Se du komitatanoj postulas sekretan voĉdonon, tiam la balotado
okazu laŭ sekreta reta voĉdono. Se sekreta reta voĉdono ne teknike
eblas, tiam la balotado okazu poŝte anstataŭ rete.

\item Gvidas retan voĉdonadon la Ĝenerala Sekretario.

\item Post elekto de nova estraro, la Komitato fiksas uzotan teknologion
por retaj voĉdon\-adoj de la periodo de ĉi tiu estraro.

\item La proceduro komencas per la ricevo de la propono far la Ĝenerala
Sekretario. Plej malfrue unu semajno post la ricevo de la propono, la
Ĝenerala Sekretario ekas la balotadon per reta informo al la
komitatanoj.

\item La balotado daŭras 28 tagojn. La estraro anoncu la rezulton plej
malfrue unu semajnon post la fino de la balotado.

\item Reta voĉdono estas kvoruma se almenaŭ duono de la komitato
voĉdonas. Sindetenoj ankaŭ estas voĉdonoj.

\item Se plejmulto de la partoprenantoj ne sindetenas, tiu el la poro kaj
la kontraŭo kiu havas la plimulton venkas. Okaze de egaleco decidas la
Prezidanto. Se simpla plejmulto de la voĉdonantoj sindetenas, oni
konsideras tion decido ne voĉdoni retpoŝte pri tiu propono.
\end{enumerate*}

\subsection{Postuloj pri reta voĉdonsistemo} \label{postuloj}

(De la laborgrupo pri reta voĉdonsistemo dum la Komitatkunveno 2008,
akceptita de la Komitato.)

\begin{enumerate*}
\item Nur la Ĝenerala Sekretario povu gvidi balotojn.
\item Estu distingo inter publikaj kaj sekretaj balotoj.
  \begin{description}
  \item[Komunaj trajtoj:] ~
    \begin{enumerate*}
    \item La ĜenSek povu vidi je ĉiu tempo, kiu jam voĉdonis en ajna baloto.
    \item Ĉiuj povas vidi la suman rezulton de la baloto post la limdato.
    \item Estas du specoj de sindetenoj: aktiva sindeteno kaj malĉeesto. La
      ĜenSek sola povu vidi, kiu sindetenis ne donante voĉon. En la
      kalkulo de la rezulto, ambaŭ specoj de sindetenoj sumiĝas kaj
      nomiĝas "Sindetenoj".
    \end{enumerate*}
  \item [Publikaj balotoj:]
    Ĉiu povu vidi, kiu kiel voĉdonis, sed nur post la limdato. Dum la
    balotado, nenia informo estas publike videbla.

  \item [Sekretaj balotoj:]
    Neniu povu vidi la unuopajn voĉojn de la voĉdonintoj, sed nur la
    suman rezulton de la baloto.
  \end{description}
\item Voĉdonoj ne estu ŝanĝeblaj.
\end{enumerate*}

\section{Uzantoj}
La reta voĉdonsistemo havas liston de uzantoj.
Ĉiu uzanto havas alirnomon kaj pasvorton por povi aliri la sistemon.
Tiu pasvorto ne estas eltrovebla el la datumbazo.
Krome la uzantolisto enhavas la plenan nomon kaj la funkcion
(ekzemple la sekcion, kiu sendis la komitatanon A), kaj retpoŝtadreson
por ĉiu.

Uzanto havas unu el la jenaj roloj.
\begin{enumerate*}
\item Komitatano A aŭ B
\item Komitatano Ĉ
\item Estrarano
\item Observanto
\item ĜenSek
\end{enumerate*}
La uzantoj en roloj 1--3 estas kune ankaŭ la grupo \emph{Komitatanoj} -- tiuj
estas la uzantoj, kiuj povas voĉdoni.

(Eblus aldoni pliajn rolojn por uzo por voĉdonoj ene de komisionoj ktp.)

\subsection{Komitatano A aŭ B}
Tiuj homoj povas partopreni en ĉiu voĉdono (krom estraro-internaj voĉdonoj).

\subsection{Komitatano Ĉ}
Tiuj homoj povas partopreni en ĉiu voĉdono krom la elekto
de komitatanoj Ĉ.\footnote{Regularo de TEJO, ĉapitro 2, paragrafo 12(2):
\emph{La komitato konsistas el: [...] (iii) la komitatanoj~Ĉ, elektitaj de la komitatanoj A~kaj~B; [...]}}

\subsection{Estrarano}
La estraranoj ankaŭ estas plenaj komitatanoj, sed ili povas partopreni
nek en la elekto de komitatanoj Ĉ\footnotemark[\thefootnote]\ nek en la elekto de estraranoj.\footnote{%
  Regularo de TEJO, ĉapitro 2, paragrafo 12(4):
  \emph{La estraranoj estas elektitaj de la komitatanoj A, B kaj Ĉ.}%
}
Krome (ekster la komitata reglamento) la reta voĉdonsistemo subtenas
internajn voĉdonojn de la estraro, kie do nur estraranoj povas partopreni.
(La rezultoj tamen estas videblaj por ĉiuj uzantoj.)

\subsection{Observanto}
Observantoj ne povas voĉdoni, sed ja (kiel ĉiu alia uzanto) vidi
la rezultojn de voĉdonoj, la uzantoliston ktp.

Observantoj povas esti eksaj komitatanoj\footnote{Kiam uzanto perdas sian
  komitatanecon, ĜenSek kutime ŝanĝas ties rolon al \emph{Observanto}.}
 aŭ aliaj aktivuloj de TEJO, kiuj interesiĝas pri la reta voĉdono de TEJO.
Ekzemple, kutime la redaktoroj de gazetoj kaj la TEJO-volontulo havas tiun
rolon, se ili ne estas jam komitatanoj.

\subsection{ĜenSek}
ĜenSek estas speciala rolo (kutime nur unu uzanto), kiu administras la
sistemon. Li kutime mem ne havas voĉdonrajton. (La sama natura persono
kutime ankaŭ havas uzantokonton kun la rolo \emph{Estrarano}, kiun li uzas
por voĉdoni mem.)

ĜenSek povas:
\begin{itemize*}
\item aldoni uzantojn
\item ŝanĝi uzantajn rolojn/nomojn/funkciojn/retadresojn
\item krei kaj sendigi novan pasvorton al uzantoj (kaze ke ili
  forgesis la malnovan)
\item forigi uzantojn
\item krei novajn voĉdonojn
\item forigi (erarajn) voĉdonojn
\item vidi, kiu voĉdonis en voĉdonoj
\end{itemize*}

Sed ĜenSek ne povas:
\begin{itemize*}
\item vidi, kiel iu voĉdonis (krom en publikaj voĉdonoj post la limdato)
\item ŝanĝi voĉdonojn
\item vidi aŭ ŝanĝi pasvortojn de uzantoj
\end{itemize*}

\subsection{Aldono de uzantoj}

ĜenSek povas/devas aldoni novajn komitatanojn al la uzantolisto.
Tiucele ekzistas aparta formularo atingebla el la uzantolisto, per kiu
eblas indiki uzanto-nomon (por ensaluto), plenan nomon, retadreson,
funkcion kaj la rolon.

Eblas indiki, ke la nova komitatano estas posteulo por jam ekzistanta
komitatano.\footnote{Tio okazas kutime nur ĉe komitatanoj A, kie la landa
  sekcio nomumas novan komitatanon anstataŭ malnova.} Tiam li ricevas
ties rolon, kaj la ekzistanta komitatano iĝas \emph{observanto}.

La sistemo poste kreas hazardan pasvorton kaj sendas ĝin retpoŝte al
la nova uzanto, sen montri ĝin al ĜenSek.

Se nuntempe estas aktivaj voĉdonoj, la nova komitatano ne tuj ricevas
voĉdonrajton, krom en kazo de posteulo, tiam la nova uzanto rajtas nur
partopreni en tiuj aktivaj voĉdonoj, kie la antaŭulo ankoraŭ ne voĉdonis.

Laŭeble ĜenSek evitas ke estas aktivaj voĉdonoj dum aldoniĝas novaj
komitatanoj.



\subsection{Redakto de uzantoj}

ĜenSek povas per speciala formularoj redakti la uzantonomon aŭ aliajn
detalojn (retpoŝt\-adreson, funkcion, nomon, rolon).

En kazo de rolo-ŝanĝo ankaŭ  ŝanĝiĝas la voĉdonrajtoj por la tiam aktivaj
voĉdonoj (en ambaŭ direktoj).

Kaze de ŝanĝo de uzantonomo la sistemo kreas novan (hazardan) pasvorton
kaj sendas ĝin al la koncerna uzanto retpoŝte.

Ankaŭ eblas (sen alia ŝanĝo) krei kaj sendigi novan (hazardan) pasvorton.

Tamen neniu iel povas vidi la aktualan pasvorton aŭ meti iun specifan
pasvorton.

\subsection{Redakto de propra konto}

Ĉiu uzanto povas per speciala formularo vidi siajn
proprajn informojn: Plena nomo, retpoŝta adreso, funkcio, salutnomo,
uzanto-roloj.

Li povas redakti la retpoŝtan adreson kaj plenan nomon.\footnote{Li ne rajtas
  ŝanĝi sian propran funkcion kaj salutnomon, por eviti trompojn, kaj
  kompreneble ankaŭ ne sian rolon.}

Li povas ŝanĝi sian propran pasvorton (por tio necesas entajpi la malnovan
kaj dufoje la novan).


\subsection{Forigo de uzantoj}

ĜenSek povas per speciala formularo forigi uzanton el la sistemo.
Tamen restas ĝiaj ĝisnunaj voĉoj kaj voĉdonrajtigoj de aktivaj
kaj pasintaj voĉdonoj en la sistemo.

(Tio kutime ne necesas, tamen: Eblas simple ŝanĝi la rolon
 al \emph{observanto}.)

% --------------------------------------------------------------------

\section{Proponoj kaj voĉdonoj}

Ĉiu propono konsistas el titolo, propona teksto, limdato kaj tipo
de voĉdono (sekreta aŭ nesekreta, kaj kiu rajtas voĉdoni).

Aldone ĉeestas la informo, kiu rajtas (aŭ rajtis) voĉdoni (kaj ĉu faris),
kiom da voĉoj estis por ĉiu varianto (jes, ne, sindeteno) kaj (por publikaj
voĉdonoj) kiu voĉis kiel.

\subsection{Propono-listo}

Estas listo de proponoj, por ĉiu uzanto atingebla. La listo
enhavas en tabela formo superrigardajn informojn pri ĉiuj proponoj,
kaj malfermaj kaj fermitaj:
\begin{itemize*}
\item Titolo
\item tipo
\item tempo de malfermo kaj fermo
\item ĉu la aktuala uzanto jam voĉis (por malfermaj)
\item resumo de la rezulto (por fermitaj).
\end{itemize*}
El la listo eblas atingi la unuopajn propono-detalo-paĝojn.

\subsection{Kreado de novaj proponoj}

ĜenSek povas krei novan proponon per tiucela formularo.
Propono devas havi titolon, kiu aperas en la listo kaj devas esti unika
inter ĉiuj proponoj) kaj proponan tekston. (La teksto estas tio, kio
estas la \emph{decido}.)
Krome necesas indiki la limtempon (kie ni proponas iun tempon taŭgan
por komitataj voĉdonoj\footnote{Komitata reglamento, 5.6:
  \emph{La balotado daŭras 28 tagojn. [...]}}) kaj la tipon de voĉdono,
  kio decidas la sekretecon/publikecon kaj kiu rajtas voĉdoni. Almenaŭ la
  sekvaj tipoj eblas:
\begin{itemize*}
\item normala publika voĉdono (ĉiu komitatano povas partopreni)
\item normala sekreta voĉdono (ĉiu komitatano povas partopreni)
\item elekto de Estraro (Estraranoj ne rajtas voĉdoni, sekreta)
\item elekto de Komitatano Ĉ (nur Komitatanoj A/B rajtas voĉdoni, sekreta)
\end{itemize*}
Krome eblas krei voĉdonojn ene de la estraro.

La formularo unue montras antaŭrigardon, kaj post ties konfirmo kreas
la voĉdonon. Ni ankaŭ kreas por ĉiu voĉdonrajtanto tabeleron en la
voĉdonanto-tabelo por tiu propono, indikanta ke li ankoraŭ ne voĉis.

\subsection{Propono-detaloj} \label{propono-detaloj}

Ĉiu uzanto povas rigardi detalan paĝon de ĉiu propono.

Ĝi enhavas la jenajn informojn:
\begin{itemize*}
\item Ĉu la voĉdono estas ankoraŭ aktiva?
\item Titolo
\item Teksto de la propono
\item tempo de malfermo kaj fermo
\item nombro kaj listo de la voĉdonrajtantoj
\item Ĉu la voĉdono estas publika?
\end{itemize*}
Post la fermo de la voĉdono:
\begin{itemize*}
\item La rezulton (kiom jes/ne/sindetenoj/ne voĉis)\footnote{%
    Tie estas malgranda kontraŭdiro inter la reglamento
    (sekcio~\ref{reglamento}) kaj la \emph{postuloj} (sekcio~\ref{postuloj}):
    Laŭ la reglamento necesas scii, kiom da homoj entute voĉis, dum laŭ
    la postuloj ni kunigu la aktivajn kaj pasivajn sindetenojn en unu nombro.
    \par
    Ni decidis ke la reglamento estas pli grava, do la programo montras ĉiujn
    nombrojn.
  }
\item Se estis publika voĉdono, listoj de voĉdonantoj por la
  unuopaj variantoj.\footnote{Ni tie distingas inter \emph{aktive sindetenis}
    kaj \emph{ne voĉdonis}.}
\end{itemize*}
Se la aktuala uzanto estas ĜenSek:
\begin{itemize*}
\item Nombro kaj listo de voĉdonintoj
\item Nombro kaj listo de nevoĉdonintoj
\end{itemize*}

Antaŭ la fermo de la voĉdono la nombro de voĉoj por unuopaj variantoj,
aŭ kiel iu voĉis, ne estu montrita.

\subsection{Voĉdono}

Por ĉiu propono kaj ĉiu uzanto, kiu rajtas voĉdoni en ĝi,
estas formularo kie la uzanto povas elekti inter la jenaj
ebloj:
\begin{itemize*}
\item Jes
\item Ne
\item Sindeteno
\end{itemize*}
La formularo ankaŭ montras la tekston de la propono apud la
voĉdonilo.

La formularo estas atingebla de la detalo-paĝo
(sekcio \ref{propono-detaloj}).

La formularo nur estas montrata por uzantoj, kiuj rajtas voĉi
pri tiu propono, kaj kiuj ankoraŭ ne voĉis. La sistemo certigas
ke uzanto ne povas plurfoje voĉi.

Por uzantoj, kiuj jam voĉis, ni anstataŭe montras, kiel
tiu uzanto voĉis (kaze de publika voĉdono -- en sekreta voĉdono
ni nur montras, ke li jam voĉis).

Dum la voĉdono ni ŝanĝas la kvanton de jes/ne/sindetenoj en la
propono, la \emph{jam voĉis}-econ de la voĉdonanto, kaj registras
la voĉon en la voĉoj-listo.

\section{Administrado}

\subsection{Forigi ŝlosojn}

La sistemo uzas ŝloso-tabelon por certigi, ke ĉiam nur unu uzanto samtempe
povas uzi (kaj specife ŝanĝi) la propono-tabelon kun la voĉo-nombroj.

Kaze de teknikaj problemoj povas okazi, ke iu ŝloso ne estas forigita --
tiam neniu plu povas voĉdoni. Tiuokaze ĜenSek povas voki specialan
tiucelan paĝon por forigi la ŝlosojn.

\subsection{Instalado}

La sistemo venas kun instal-programo. Tiu povas krei la datumbazajn
tabelojn kaj krei la ĜenSek-uzanton. (Antaŭ la uzo necesas tamen
krei la datumbazon kaj enmeti la alirdatumojn al ĝi en la koncernan
konfiguran dosieron.)

Alternative la instalilo ankaŭ kapablas forigi la instalitajn tabelojn, aŭ
restarigi la pasvorton de ĜenSek (ĉar ĜenSek ja ne povas sendigi novan
pasvorton al si, se li ne plu havas aliron pro forgeso).


\section{Sekurecaj konsideroj}

Por homoj (aŭ komputilaj programoj), kiuj havas rektan aliron
al la datumbazo, ne eblas ekscii pasvortojn de uzantoj. Sed ja eblas
ŝanĝi la pasvorton al iu dezirata. %\enlargethispage{\baselineskip}

Ankaŭ eblas ekscii, kiu voĉas kiel (en publikaj balotoj) aŭ kiel
estas la aktuala stato de la voĉdono (en ĉiuj balotoj) antaŭ la fino
de la voĉdontempo.

Eĉ pli, eblas ŝanĝi la rezulton de voĉdonoj, kaj
en sekreta voĉdono ne ekzistas eblo rimarki tion. (En publika voĉdono
la partoprenantoj mem povas kontroli, ĉu ilia propra voĉo estas
ĝuste registrita, kaj iu povas rekalkuli la sumon.)

Do, la sekurecaj postuloj en sekcio \ref{postuloj} nur estas validaj, se
oni fidas la servilajn administrantojn (kaj ĉiujn, kiuj povas aliri la
datumbazon).


Tion ne eblas facile eviti sen pli larĝa uzo de kriptografio -- eblan
vojon mi skizis en la dokumento \emph{Pripensoj pri la reta voĉdonsistemo}.



\end{document}
